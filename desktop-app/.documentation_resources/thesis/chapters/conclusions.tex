\chapter*{Concluzii}
\addcontentsline{toc}{chapter}{Concluzii}

\textit{Nota: Acest capitol va fi finalizat la sfârșitul proiectului, după implementarea și validarea tuturor funcționalităților planificate.}

\section*{Sinteza realizărilor}

\textit{[To be completed]}

Lucrarea de față și-a propus dezvoltarea platformei LiveCode, o soluție modernă pentru gestionarea și colaborarea în timp real asupra proiectelor remote, adresând limitările soluțiilor existente precum WinSCP și FileZilla.

Principalele realizări ale proiectului includ:

\begin{enumerate}
    \item \textbf{Analiză comprehensivă} a problemelor existente în tooling-ul actual de management remote, identificând necesitatea unui sistem de file locking distribuit și a unei interfețe moderne

    \item \textbf{Proiectare arhitecturală} solidă bazată pe tehnologii moderne:
        \begin{itemize}
            \item Tauri v2 pentru aplicația desktop cu footprint redus
            \item React 19 pentru interfață utilizator reactivă
            \item Rust pentru backend performant și sigur
            \item PostgreSQL pentru persistență robustă
        \end{itemize}

    \item \textbf{Implementarea funcționalităților core:}
        \begin{itemize}
            \item Sistem de autentificare securizat cu criptare end-to-end
            \item Gestionare conexiuni SSH/SFTP cu multiple metode de autentificare
            \item File browser dual-pane cu operațiuni complete
            \item Sistem de file locking distribuit pentru prevenirea conflictelor
            \item Notificări în timp real prin WebSocket
        \end{itemize}

    \item \textbf{Validare} prin teste și exemple de utilizare din scenarii reale
\end{enumerate}

\section*{Provocări tehnice și soluții}

\textit{[To be completed based on implementation challenges]}

Pe parcursul dezvoltării platformei LiveCode, au fost întâmpinate și rezolvate mai multe provocări tehnice semnificative:

\begin{itemize}
    \item \textbf{Sincronizarea lock-urilor distribuite}
        \begin{itemize}
            \item Provocare: Asigurarea consistenței lock-urilor între multiple instanțe
            \item Soluție: Implementare heartbeat mechanism cu timeout automat și reconciliation
        \end{itemize}

    \item \textbf{Performanța transferurilor SFTP}
        \begin{itemize}
            \item Provocare: Menținerea vitezelor competitive cu soluții mature
            \item Soluție: Connection pooling, buffering optimizat, async I/O
        \end{itemize}

    \item \textbf{Securitatea credențialelor}
        \begin{itemize}
            \item Provocare: Stocare sigură fără a compromite UX
            \item Soluție: Criptare AES-256-GCM cu master key derivat din parola user
        \end{itemize}

    \item \textbf{Cross-platform compatibility}
        \begin{itemize}
            \item Provocare: Diferențe între Windows, macOS și Linux
            \item Soluție: Abstracțiuni oferite de Tauri și testare pe toate platformele
        \end{itemize}
\end{itemize}

\section*{Contribuții și valoare adăugată}

Comparativ cu soluțiile existente, LiveCode aduce următoarele contribuții:

\begin{enumerate}
    \item \textbf{Inovație tehnică:} Prima aplicație desktop SFTP bazată pe Tauri v2, demonstrând viabilitatea framework-ului pentru aplicații complexe de sistem

    \item \textbf{File locking distribuit:} Mecanism unic în categoria tooling-ului SFTP desktop, rezolvând problema editărilor concurente

    \item \textbf{Performanță superioară:} Consum redus de resurse (RAM, dimensiune aplicație) comparativ cu soluții bazate pe Electron

    \item \textbf{Developer experience:} Interfață modernă, notificări în timp real, workflow optimizat pentru colaborare

    \item \textbf{Open-source:} Cod disponibil comunității pentru extindere și customizare
\end{enumerate}

\section*{Limitări actuale}

\textit{[To be completed - honest assessment of current limitations]}

În stadiul actual de dezvoltare, LiveCode prezintă următoarele limitări care vor fi adresate în versiuni viitoare:

\begin{itemize}
    \item Suport limitat pentru protocoale suplimentare (FTP, WebDAV)
    \item Lipsă integrare cu sisteme de version control (Git)
    \item Funcționalități de team management incomplete
    \item Coverage incomplet de teste automate
    \item Documentație de utilizare în dezvoltare
\end{itemize}

\section*{Direcții de dezvoltare viitoare}

\textit{[To be completed - realistic future roadmap]}

Dezvoltarea LiveCode va continua în următoarele direcții:

\subsection*{Pe termen scurt (3-6 luni)}
\begin{itemize}
    \item Finalizare sistem de file locking cu toate edge cases
    \item Implementare diff viewer integrat pentru rezolvare conflicte
    \item Adăugare suport pentru file templates și snippets
    \item Extindere suite de teste automate
    \item Optimizare performanță transferuri mari
\end{itemize}

\subsection*{Pe termen mediu (6-12 luni)}
\begin{itemize}
    \item Sistem de team management și permisiuni granulare
    \item Integrare Git pentru sincronizare automată
    \item Plugin system pentru extensibilitate
    \item Mobile companion app pentru monitoring
    \item Cloud sync pentru configurații între dispozitive
\end{itemize}

\subsection*{Pe termen lung (12+ luni)}
\begin{itemize}
    \item Suport protocoale suplimentare (WebDAV, S3, Azure Blob)
    \item Live coding collaboration (multiplayer editing)
    \item AI-assisted code suggestions și auto-completion pentru fișiere remote
    \item Marketplace pentru plugins și teme
    \item Enterprise features (SSO, audit logging, compliance)
\end{itemize}

\section*{Impactul educațional și profesional}

\textit{[To be completed - personal reflections]}

Dezvoltarea platformei LiveCode a reprezentat o experiență de învățare cuprinzătoare, acoperind multiple domenii:

\begin{itemize}
    \item \textbf{Programare systems-level:} Înțelegere profundă a Rust, memory safety, ownership
    \item \textbf{Arhitectură software:} Proiectare layered architecture, separation of concerns
    \item \textbf{Securitate:} Criptografie aplicată, threat modeling, secure coding practices
    \item \textbf{Protocoale de rețea:} SSH, SFTP, WebSocket, flow control
    \item \textbf{Baze de date:} Schema design, migrations, query optimization
    \item \textbf{UI/UX design:} React patterns, responsive design, accessibility
    \item \textbf{DevOps:} Build systems, CI/CD, cross-platform distribution
\end{itemize}

Competențele dobândite sunt direct aplicabile în industria software modernă, unde:
\begin{itemize}
    \item Rust devine din ce în ce mai adoptat pentru aplicații performante și sigure
    \item Desktop applications rămân relevante pentru tooling profesional
    \item Real-time collaboration devine standard în developer tools
    \item Security și privacy sunt priorități critice
\end{itemize}

\section*{Cuvânt final}

\textit{[To be completed]}

LiveCode demonstrează că este posibil să construiești tooling modern, performant și user-friendly folosind tehnologii de ultimă generație. Proiectul nu doar că rezolvă probleme practice din workflow-ul dezvoltatorilor, dar servește și ca proof-of-concept pentru viabilitatea ecosistemului Rust în aplicații desktop complexe.

Deși dezvoltarea continuă, fundamentele solide puse în această lucrare de licență asigură o bază solidă pentru evoluția platformei către un tool profesional de producție.

Mulțumiri speciale coordonatorului științific, \textbf{Conf. Dr. Zalinescu Adrian}, pentru îndrumare și suport pe parcursul acestui proiect.

\vspace{1cm}

\begin{flushright}
Marcoci Fabian-Constantin \\
Iași, 2026
\end{flushright}

\newpage