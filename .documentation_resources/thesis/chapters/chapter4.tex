\chapter{Exemple de utilizare}

\textit{Nota: Acest capitol va fi dezvoltat pe măsură ce funcționalitățile platformei sunt finalizate și testate. Vor fi adăugate capturi de ecran, demonstrații de workflow și scenarii de utilizare reale.}

\section{Configurarea inițială}

\textit{[PLANNED]}

\subsection{Instalarea aplicației}

Procesul de instalare va include:
\begin{itemize}
    \item Download installer pentru Windows/macOS/Linux
    \item Instalare și configurare inițială
    \item Verificare integritate și semnături digitale
\end{itemize}

\subsection{Crearea primului cont}

Workflow de onboarding pentru noi utilizatori:
\begin{enumerate}
    \item Accesare ecran de înregistrare
    \item Completare formular (email, parolă)
    \item Validare email (opțional)
    \item Login și configurare profil
\end{enumerate}

\section{Gestionarea conexiunilor SSH/SFTP}

\textit{[PLANNED]}

\subsection{Adăugarea unei conexiuni noi}

Demonstrație pas cu pas:
\begin{enumerate}
    \item Navigare la secțiunea "Connections"
    \item Click pe "Add New Connection"
    \item Completare detalii conexiune:
        \begin{itemize}
            \item Host: example.com
            \item Port: 22
            \item Username: developer
            \item Tip autentificare: SSH Key
        \end{itemize}
    \item Testare conexiune
    \item Salvare configurație
\end{enumerate}

\textit{[Aici va fi adăugată captură de ecran cu formularul de adăugare conexiune]}

\subsection{Tipuri de autentificare}

\subsubsection{Autentificare cu parolă}

Exemplu de configurare:
\begin{itemize}
    \item Selectare "Password Authentication"
    \item Introducere parolă (salvată criptat)
    \item Opțiune "Remember password"
\end{itemize}

\subsubsection{Autentificare cu cheie SSH}

Workflow pentru chei SSH:
\begin{itemize}
    \item Selectare fișier cheie privată (id\_rsa, id\_ed25519)
    \item Introducere passphrase (dacă există)
    \item Validare format cheie
    \item Test autentificare
\end{itemize}

\section{Browsing și operațiuni pe fișiere}

\textit{[PLANNED]}

\subsection{Navigarea în sistemul de fișiere remote}

Interfața dual-pane:
\begin{itemize}
    \item Panel stânga - Fișiere locale
    \item Panel dreapta - Fișiere remote
    \item Sincronizare navigare între paneluri
\end{itemize}

\textit{[Aici va fi adăugată captură de ecran cu file browser dual-pane]}

\subsection{Upload și download de fișiere}

Scenarii tipice:

\textbf{Scenariul 1: Upload fișier de configurare}
\begin{enumerate}
    \item Selectare fișier local: \texttt{config/app.conf}
    \item Drag \& drop către panel remote
    \item Monitorizare progress bar
    \item Confirmare transfer reușit
\end{enumerate}

\textbf{Scenariul 2: Download log files pentru debugging}
\begin{enumerate}
    \item Navigare la \texttt{/var/log/application/}
    \item Selectare multiplă fișiere log
    \item Click dreapta → Download
    \item Selectare destinație locală
    \item Monitorizare transfer batch
\end{enumerate}

\subsection{Operațiuni avansate}

\textit{[PLANNED]}

\begin{itemize}
    \item Modificare permisiuni (chmod)
    \item Redenumire fișiere/directoare
    \item Creare directoare noi
    \item Ștergere cu confirmare
    \item Copy/Move între directoare remote
\end{itemize}

\section{Sistemul de file locking în acțiune}

\textit{[PLANNED - Funcționalitate core]}

\subsection{Scenariul 1: Editare fără conflict}

Workflow ideal:
\begin{enumerate}
    \item \textbf{User A} deschide \texttt{server.conf} pentru editare
    \item Sistemul:
        \begin{itemize}
            \item Verifică disponibilitatea fișierului
            \item Creează lock în database
            \item Notifică toți utilizatorii conectați
            \item Deschide editorul integrat
        \end{itemize}
    \item \textbf{User B} încearcă să deschidă același fișier
    \item Sistemul afișează notificare:
        \begin{quote}
        "Fișierul server.conf este în curs de editare de către User A (john@example.com). Doriți să deschideți în modul read-only?"
        \end{quote}
    \item \textbf{User A} finalizează modificările și salvează
    \item Sistemul eliberează lock-ul
    \item \textbf{User B} primește notificare: "server.conf este acum disponibil"
\end{enumerate}

\textit{[Aici vor fi adăugate capturi de ecran pentru fiecare pas]}

\subsection{Scenariul 2: Lock timeout și recuperare}

Gestionarea lock-urilor abandonate:
\begin{enumerate}
    \item User deschide fișier și își închide aplicația fără să salveze
    \item Lock rămâne activ pentru 5 minute (grace period)
    \item După timeout, sistemul:
        \begin{itemize}
            \item Eliberează automat lock-ul
            \item Notifică utilizatorii în așteptare
            \item Loghează evenimentul
        \end{itemize}
\end{enumerate}

\subsection{Scenariul 3: Force unlock (administrator)}

Pentru situații excepționale:
\begin{enumerate}
    \item Administrator identifică lock blocat
    \item Click dreapta pe fișier → "View Lock Info"
    \item Verificare detalii lock (cine, când, de cât timp)
    \item Click "Force Unlock" cu confirmare
    \item Sistem notifică utilizatorul care deținea lock-ul
\end{enumerate}

\section{Colaborarea în timp real}

\textit{[PLANNED]}

\subsection{Notificări live}

Tipuri de notificări demonstrate:
\begin{itemize}
    \item File locked/unlocked events
    \item User online/offline status
    \item Modificări în directoare monitorizate
    \item System events (connection lost, reconnected)
\end{itemize}

\subsection{Activity feed}

Timeline cu activitatea echipei:
\begin{itemize}
    \item "John locked database.sql - 2 minutes ago"
    \item "Maria uploaded deploy.sh - 5 minutes ago"
    \item "Alex unlocked config.yml - 10 minutes ago"
\end{itemize}

\section{Workflow complet: De la conectare la deployment}

\textit{[PLANNED]}

\subsection{Cazul de utilizare: Update configurație server web}

Scenariul real și complet:

\textbf{Context:} Echipă de 3 developeri lucrează pe același server de staging, trebuie să actualizeze configurația Nginx pentru un nou endpoint API.

\textbf{Pași:}
\begin{enumerate}
    \item \textbf{Connect:}
        \begin{itemize}
            \item Developer 1 deschide LiveCode
            \item Selectează conexiunea "Staging Server"
            \item Autentificare automată cu cheie SSH salvată
            \item Conectare stabilită
        \end{itemize}

    \item \textbf{Navigate:}
        \begin{itemize}
            \item Navigare la \texttt{/etc/nginx/sites-available/}
            \item Sortare fișiere după data modificării
            \item Identificare fișier: \texttt{api.example.com.conf}
        \end{itemize}

    \item \textbf{Lock \& Edit:}
        \begin{itemize}
            \item Double-click pe fișier
            \item Sistem verifică lock status
            \item Fișier disponibil → lock acquired
            \item Developer 2 și 3 primesc notificare
            \item Editor se deschide cu conținut
        \end{itemize}

    \item \textbf{Modify:}
        \begin{itemize}
            \item Adăugare bloc location pentru \texttt{/api/v2/}
            \item Syntax highlighting pentru Nginx config
            \item Auto-save draft local (fără unlock)
        \end{itemize}

    \item \textbf{Save \& Deploy:}
        \begin{itemize}
            \item Click "Save" → upload la server
            \item Lock eliberat automat
            \item Notificare către echipă: "api.example.com.conf updated"
            \item Developer verifică în terminal: \texttt{nginx -t}
            \item Reload Nginx: \texttt{systemctl reload nginx}
        \end{itemize}
\end{enumerate}

\textit{[Aici va fi adăugat un flow diagram vizual]}

\section{Performanță și metrici}

\textit{[PLANNED]}

\subsection{Benchmark-uri}

Comparații de performanță cu WinSCP:

\begin{table}[h]
\caption{Transfer speed comparison}
\centering
\begin{tabular}{|l|c|c|}
\hline
\textbf{Operațiune} & \textbf{WinSCP} & \textbf{LiveCode} \\
\hline
Upload 1GB file & X sec & Y sec \\
Download 1000 small files & X sec & Y sec \\
Directory listing (10k files) & X sec & Y sec \\
\hline
\end{tabular}
\end{table}

\textit{[Benchmark-urile vor fi efectuate după implementarea completă]}

\subsection{Consumul de resurse}

Monitorizare:
\begin{itemize}
    \item RAM usage: idle vs active transfer
    \item CPU usage: during file operations
    \item Disk I/O: sustained transfer rates
    \item Network efficiency: protocol overhead
\end{itemize}

\section{Gestionarea erorilor}

\textit{[PLANNED]}

\subsection{Scenarii de eroare și recuperare}

\textbf{Eroare 1: Connection timeout}
\begin{itemize}
    \item Cauză: Server inaccesibil sau firewall
    \item Notificare user: "Failed to connect to example.com:22"
    \item Acțiuni disponibile: Retry, Edit connection, Cancel
\end{itemize}

\textbf{Eroare 2: Authentication failed}
\begin{itemize}
    \item Cauză: Credențiale invalide sau cheie expirată
    \item Notificare: "Authentication failed. Please check credentials"
    \item Acțiuni: Re-enter password, Select different key
\end{itemize}

\textbf{Eroare 3: Permission denied}
\begin{itemize}
    \item Cauză: Permisiuni insuficiente pe server
    \item Notificare: "Permission denied for /var/www/html/"
    \item Sugestie: "Contact administrator or check file permissions"
\end{itemize}

\section{Concluzii parțiale}

Acest capitol a demonstrat (și va demonstra pe măsură ce funcționalitățile sunt implementate) cum LiveCode îmbunătățește workflow-ul de lucru cu servere remote prin:

\begin{itemize}
    \item Interfață intuitivă și modernă
    \item Sistem de file locking transparent și eficient
    \item Notificări în timp real pentru colaborare
    \item Performanță optimizată pentru operațiuni frecvente
    \item Gestionare elegantă a erorilor
\end{itemize}

Exemplele prezentate ilustrează cazuri de utilizare reale din activitatea zilnică a dezvoltatorilor și administratorilor de sistem.

\newpage