\chapter*{Introducere}
\addcontentsline{toc}{chapter}{Introducere}

În era digitală actuală, colaborarea la distanță asupra proiectelor software a devenit o necesitate fundamentală pentru echipele de dezvoltare. Gestionarea eficientă a fișierelor remote, editarea colaborativă și sincronizarea modificărilor reprezintă provocări constante pentru dezvoltatori, administratori de sistem și profesioniști IT.

WinSCP, unul dintre cele mai utilizate instrumente pentru transferul de fișiere prin SSH/SFTP, a deservit comunitatea tehnică timp de peste două decenii. Cu toate acestea, în contextul nevoilor moderne de colaborare în timp real, limitările sale devin din ce în ce mai evidente: lipsa suportului pentru colaborare simultană, imposibilitatea de a preveni conflictele de editare și interfața care nu reflectă standardele contemporane de user experience.

\section*{Contextul proiectului}

LiveCode își propune să răspundă acestor provocări prin dezvoltarea unei platforme desktop moderne, construite cu tehnologii de ultimă generație: Tauri v2 pentru aplicația desktop, React 19 pentru interfața utilizator, Rust pentru logica de business și PostgreSQL pentru persistența datelor. Această combinație tehnologică asigură nu doar performanță superioară și securitate sporită, ci și o experiență de utilizare fluidă și intuitivă.

Proiectul adresează o problemă reală și actuală: necesitatea unei soluții moderne pentru gestionarea colaborativă a proiectelor remote, care să integreze mecanisme de blocare a fișierelor (file locking) pentru prevenirea conflictelor de editare simultană.

\section*{Obiectivele lucrării}

Prezenta lucrare de licență își propune să documenteze procesul de proiectare, dezvoltare și implementare a platformei LiveCode, având următoarele obiective principale:

\begin{enumerate}
    \item \textbf{Analiza problemei} - Identificarea limitărilor soluțiilor existente și definirea cerințelor pentru o platformă modernă de colaborare
    \item \textbf{Proiectarea arhitecturală} - Elaborarea unei arhitecturi robuste, scalabile și securizate, bazată pe principii moderne de software engineering
    \item \textbf{Implementarea soluției} - Dezvoltarea efectivă a platformei, integrând tehnologii de ultimă generație
    \item \textbf{Validarea funcționalității} - Demonstrarea capabilităților platformei prin exemple concrete de utilizare
\end{enumerate}

\section*{Structura lucrării}

Lucrarea este organizată în patru capitole principale, fiecare abordând aspecte distincte ale proiectului:

\textbf{Capitolul 1 - Analiza Problemei} examinează provocările actuale în gestionarea colaborativă a proiectelor remote, analizează soluțiile existente și definește cerințele pentru platforma LiveCode.

\textbf{Capitolul 2 - Tehnologii și Arhitectură} prezintă stack-ul tehnologic ales (Tauri, React, Rust, PostgreSQL), justifică deciziile arhitecturale și descrie modul de integrare a componentelor.

\textbf{Capitolul 3 - Implementare} detaliază procesul de dezvoltare, abordând aspecte precum autentificarea utilizatorilor, gestionarea conexiunilor SSH/SFTP, implementarea mecanismului de file locking și alte funcționalități esențiale.

\textbf{Capitolul 4 - Exemple de Utilizare} demonstrează funcționalitatea platformei prin scenarii reale de utilizare, ilustrând fluxurile principale și beneficiile aduse utilizatorilor.

\section*{Contribuții}

Principalele contribuții ale acestei lucrări includ:

\begin{itemize}
    \item Proiectarea și implementarea unei arhitecturi moderne pentru aplicații desktop cross-platform folosind Tauri v2
    \item Dezvoltarea unui mecanism de file locking distribuit pentru prevenirea conflictelor de editare
    \item Integrarea securizată a conexiunilor SSH/SFTP într-o aplicație desktop modernă
    \item Crearea unei interfețe utilizator intuitive și responsive folosind React 19
    \item Implementarea unui sistem de autentificare și gestionare a sesiunilor utilizând Rust și PostgreSQL
\end{itemize}

LiveCode nu reprezintă doar o alternativă modernă la WinSCP, ci o reimaginare completă a modului în care dezvoltatorii pot colabora eficient asupra proiectelor remote, aducând colaborarea în timp real în centrul experienței de lucru cu fișiere remote.

\newpage