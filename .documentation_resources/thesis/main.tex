% Universitatea de Vest din Timișoara
% Facultatea de Matematică și Informatică
% Lucrare de Licență - LiveCode
% Autor: Marcoci Fabian-Constantin
% Coordonator: Zalinescu Adrian

\documentclass[12pt,a4paper]{report}

% Pachete necesare
\usepackage[utf8]{inputenc}
\usepackage[romanian]{babel}
\usepackage{graphicx}
\usepackage{hyperref}
\usepackage{listings}
\usepackage{xcolor}
\usepackage{geometry}
\usepackage{fancyhdr}
\usepackage{titlesec}
\usepackage{tocloft}
\usepackage{amsmath}
\usepackage{amssymb}
\usepackage{caption}
\usepackage{subcaption}

% Configurare pagină
\geometry{
    a4paper,
    left=3cm,
    right=2cm,
    top=2.5cm,
    bottom=2.5cm
}

% Configurare hyperlink-uri
\hypersetup{
    colorlinks=true,
    linkcolor=black,
    filecolor=black,
    urlcolor=blue,
    citecolor=black
}

% Configurare listări cod
\lstset{
    basicstyle=\ttfamily\small,
    breaklines=true,
    frame=single,
    numbers=left,
    numberstyle=\tiny,
    keywordstyle=\color{blue},
    commentstyle=\color{green!60!black},
    stringstyle=\color{red},
    showstringspaces=false
}

% Configurare stiluri capitole
\titleformat{\chapter}[display]
{\normalfont\huge\bfseries}{\chaptertitlename\ \thechapter}{20pt}{\Huge}
\titlespacing*{\chapter}{0pt}{0pt}{40pt}

% Header și footer
\pagestyle{fancy}
\fancyhf{}
\fancyhead[R]{\thepage}
\fancyhead[L]{\leftmark}
\renewcommand{\headrulewidth}{0.4pt}

\begin{document}

% ============================================
% FRONT MATTER
% ============================================

\pagenumbering{roman}

% Copertă
\begin{titlepage}
    \centering

    \vspace*{1cm}

    {\Large UNIVERSITATEA ALEXANDRU IOAN CUZA DIN IAȘI}

    \vspace{0.5cm}

    {\Large FACULTATEA DE INFORMATICĂ}

    \vspace{2cm}

    % Logo universitate (opțional)
    % \includegraphics[width=0.3\textwidth]{images/logo_uvt.png}

    \vspace{2cm}

    {\huge\bfseries LUCRARE DE LICENȚĂ}

    \vspace{2cm}

    {\LARGE\bfseries LiveCode}

    \vspace{0.5cm}

    {\large Platformă desktop pentru gestionarea și colaborarea în timp real asupra proiectelor remote}

    \vfill

    \begin{flushleft}
        {\large
        \textbf{Absolvent:} Marcoci Fabian-Constantin \\[0.3cm]
        \textbf{Coordonator științific:} Conf. Dr. Zalinescu Adrian
        }
    \end{flushleft}

    \vspace{1cm}

    {\large Iași, 2026}

\end{titlepage}

\newpage

% Pagina de titlu
\begin{titlepage}
    \centering

    \vspace*{2cm}

    {\Large UNIVERSITATEA ALEXANDRU IOAN CUZA DIN IAȘI}

    {\Large FACULTATEA DE INFORMATICĂ}

    \vspace{4cm}

    {\Huge\bfseries LiveCode}

    \vspace{3cm}

    {\large Absolvent}

    {\Large\bfseries Marcoci Fabian-Constantin}

    \vspace{2cm}

    {\large Sesiunea: Iulie, 2026}

    \vspace{3cm}

    {\large Coordonator științific}

    {\Large\bfseries Conf. Dr. Zalinescu Adrian}

    \vfill

\end{titlepage}

\newpage

% Declarații
\chapter*{Declarații}
\addcontentsline{toc}{chapter}{Declarații}

\section*{Declarație de originalitate}

Subsemnatul \textbf{Marcoci Fabian-Constantin}, CNP 5020825226751, domiciliat în România, Județul Iași, Municipiul Iași, Șos. Nicolina nr. 35, bl. 968, tr. 3, et. 2, ap. 5, cod poștal 700688, absolvent al Universității Alexandru Ioan Cuza din Iași, Facultatea de Informatică, specializarea Informatică, promoția 2026, declar pe propria răspundere, cunoscând consecințele falsului în declarații în sensul art. 326 din Noul Cod Penal și dispozițiile Legii Educației Naționale nr. 1/2011 art. 143 al. 4 și 5 referitoare la plagiat, că lucrarea de licență cu titlul:

\begin{center}
    \textbf{LiveCode - Platformă desktop pentru gestionarea și colaborarea în timp real asupra proiectelor remote}
\end{center}

elaborată sub îndrumarea domnului \textbf{Conf. Dr. Zalinescu Adrian}, pe care urmează să o susțin în fața comisiei este originală, îmi aparține și îmi asum conținutul său în întregime.

De asemenea, declar că sunt de acord ca lucrarea mea de licență să fie verificată prin orice modalitate legală pentru confirmarea originalității, consimțind inclusiv la introducerea conținutului său într-o bază de date în acest scop.

Am luat la cunoștință despre faptul că este interzisă comercializarea de lucrări științifice în vederea facilitării falsificării de către cumpărător a calității de autor al unei lucrări de licență, de diplomă sau de disertație și în acest sens, declar pe proprie răspundere că lucrarea de față nu a fost copiată ci reprezintă rodul cercetării pe care am întreprins-o.

\vspace{2cm}

\noindent
Data: \underline{\hspace{4cm}} \hfill Semnătura: \underline{\hspace{4cm}}

\newpage

\section*{Declarație de consimțământ}

Prin prezenta declar că sunt de acord ca Lucrarea de licență cu titlul \textbf{LiveCode - Platformă desktop pentru gestionarea și colaborarea în timp real asupra proiectelor remote}, codul sursă al programelor și celelalte conținuturi (grafice, multimedia, date de test etc.) care însoțesc această lucrare să fie utilizate în cadrul Facultății de Informatică.

De asemenea, sunt de acord ca Facultatea de Informatică de la Universitatea Alexandru Ioan Cuza din Iași, să utilizeze, modifice, reproducă și să distribuie în scopuri necomerciale programele-calculator, format executabil și sursă, realizate de mine în cadrul prezentei lucrări de licență.

\vspace{2cm}

\noindent
Iași, data \underline{\hspace{3cm}}

\vspace{1cm}

\noindent
Absolvent Marcoci Fabian-Constantin

\noindent
\underline{\hspace{6cm}}

(semnătura în original)

\newpage

% Cuprins
\tableofcontents
\newpage

% ============================================
% MAIN MATTER
% ============================================

\pagenumbering{arabic}
\setcounter{page}{1}

% Introducere
\chapter*{Introducere}
\addcontentsline{toc}{chapter}{Introducere}

În era digitală actuală, colaborarea la distanță asupra proiectelor software a devenit o necesitate fundamentală pentru echipele de dezvoltare. Gestionarea eficientă a fișierelor remote, editarea colaborativă și sincronizarea modificărilor reprezintă provocări constante pentru dezvoltatori, administratori de sistem și profesioniști IT.

WinSCP, unul dintre cele mai utilizate instrumente pentru transferul de fișiere prin SSH/SFTP, a deservit comunitatea tehnică timp de peste două decenii. Cu toate acestea, în contextul nevoilor moderne de colaborare în timp real, limitările sale devin din ce în ce mai evidente: lipsa suportului pentru colaborare simultană, imposibilitatea de a preveni conflictele de editare și interfața care nu reflectă standardele contemporane de user experience.

\section*{Contextul proiectului}

LiveCode își propune să răspundă acestor provocări prin dezvoltarea unei platforme desktop moderne, construite cu tehnologii de ultimă generație: Tauri v2 pentru aplicația desktop, React 19 pentru interfața utilizator, Rust pentru logica de business și PostgreSQL pentru persistența datelor. Această combinație tehnologică asigură nu doar performanță superioară și securitate sporită, ci și o experiență de utilizare fluidă și intuitivă.

Proiectul adresează o problemă reală și actuală: necesitatea unei soluții moderne pentru gestionarea colaborativă a proiectelor remote, care să integreze mecanisme de blocare a fișierelor (file locking) pentru prevenirea conflictelor de editare simultană.

\section*{Obiectivele lucrării}

Prezenta lucrare de licență își propune să documenteze procesul de proiectare, dezvoltare și implementare a platformei LiveCode, având următoarele obiective principale:

\begin{enumerate}
    \item \textbf{Analiza problemei} - Identificarea limitărilor soluțiilor existente și definirea cerințelor pentru o platformă modernă de colaborare
    \item \textbf{Proiectarea arhitecturală} - Elaborarea unei arhitecturi robuste, scalabile și securizate, bazată pe principii moderne de software engineering
    \item \textbf{Implementarea soluției} - Dezvoltarea efectivă a platformei, integrând tehnologii de ultimă generație
    \item \textbf{Validarea funcționalității} - Demonstrarea capabilităților platformei prin exemple concrete de utilizare
\end{enumerate}

\section*{Structura lucrării}

Lucrarea este organizată în patru capitole principale, fiecare abordând aspecte distincte ale proiectului:

\textbf{Capitolul 1 - Analiza Problemei} examinează provocările actuale în gestionarea colaborativă a proiectelor remote, analizează soluțiile existente și definește cerințele pentru platforma LiveCode.

\textbf{Capitolul 2 - Tehnologii și Arhitectură} prezintă stack-ul tehnologic ales (Tauri, React, Rust, PostgreSQL), justifică deciziile arhitecturale și descrie modul de integrare a componentelor.

\textbf{Capitolul 3 - Implementare} detaliază procesul de dezvoltare, abordând aspecte precum autentificarea utilizatorilor, gestionarea conexiunilor SSH/SFTP, implementarea mecanismului de file locking și alte funcționalități esențiale.

\textbf{Capitolul 4 - Exemple de Utilizare} demonstrează funcționalitatea platformei prin scenarii reale de utilizare, ilustrând fluxurile principale și beneficiile aduse utilizatorilor.

\section*{Contribuții}

Principalele contribuții ale acestei lucrări includ:

\begin{itemize}
    \item Proiectarea și implementarea unei arhitecturi moderne pentru aplicații desktop cross-platform folosind Tauri v2
    \item Dezvoltarea unui mecanism de file locking distribuit pentru prevenirea conflictelor de editare
    \item Integrarea securizată a conexiunilor SSH/SFTP într-o aplicație desktop modernă
    \item Crearea unei interfețe utilizator intuitive și responsive folosind React 19
    \item Implementarea unui sistem de autentificare și gestionare a sesiunilor utilizând Rust și PostgreSQL
\end{itemize}

LiveCode nu reprezintă doar o alternativă modernă la WinSCP, ci o reimaginare completă a modului în care dezvoltatorii pot colabora eficient asupra proiectelor remote, aducând colaborarea în timp real în centrul experienței de lucru cu fișiere remote.

\newpage

% Motivație
\chapter*{Motivație}
\addcontentsline{toc}{chapter}{Motivație}

Alegerea temei acestei lucrări de licență a fost ghidată de confluența dintre nevoile practice ale dezvoltatorilor moderni și oportunitățile oferite de tehnologiile emergente în domeniul aplicațiilor desktop cross-platform.

\section*{Motivația personală}

Experiența personală de lucru cu diverse instrumente de gestionare a fișierelor remote, în special WinSCP, a evidențiat o serie de limitări care afectează productivitatea în contextul colaborării moderne. Frustrarea cauzată de conflictele de editare simultană, lipsa vizibilității asupra activității colegilor de echipă și interfața învechită au constituit punctul de plecare pentru conceptualizarea platformei LiveCode.

Pasiunea pentru tehnologiile moderne, în special Rust și Tauri, a reprezentat un factor motivațional suplimentar. Rust oferă garanții de siguranță a memoriei și performanță excepțională, în timp ce Tauri permite crearea de aplicații desktop native folosind tehnologii web, reducând semnificativ dimensiunea aplicației finale comparativ cu alternative precum Electron.

\section*{Relevanța în contextul actual}

Pandemia COVID-19 a accelerat dramatic tranziția către munca la distanță, transformând colaborarea remote dintr-o opțiune într-o necesitate. Statisticile recente arată că:

\begin{itemize}
    \item Peste 70\% dintre dezvoltatorii software lucrează cel puțin parțial de la distanță
    \item Numărul proiectelor open-source cu contributori distribuiți geografic a crescut cu peste 150\% în ultimii 3 ani
    \item Conflictele de editare simultană reprezintă una dintre principalele cauze de pierdere a timpului în echipele distribuite
\end{itemize}

În acest context, nevoia pentru instrumente moderne de colaborare asupra fișierelor remote nu a fost niciodată mai acută. LiveCode răspunde acestei nevoi prin integrarea mecanismelor de file locking direct în fluxul de lucru, permițând echipelor să colaboreze eficient fără riscul conflictelor.

\section*{Oportunitatea tehnologică}

Ecosistemul Rust a cunoscut o creștere exponențială în ultimii ani, cu adopție crescândă în industrie (Mozilla, Microsoft, Amazon, Discord). Tauri, ca framework pentru aplicații desktop bazat pe Rust, a atins maturitatea necesară pentru dezvoltarea de aplicații production-ready, oferind:

\begin{itemize}
    \item Dimensiuni reduse ale aplicației (sub 10 MB comparativ cu 100+ MB pentru Electron)
    \item Consum redus de memorie RAM (sub 50 MB vs 300+ MB pentru Electron)
    \item Securitate sporită prin izolarea proceselor și verificările de tip din Rust
    \item Performanță nativă pe toate platformele majore (Windows, macOS, Linux)
\end{itemize}

\section*{Valoarea educațională}

Din perspectiva formării profesionale, acest proiect oferă oportunitatea de a:

\begin{enumerate}
    \item \textbf{Aprofunda cunoștințele de programare sistem} prin utilizarea Rust pentru componente critice de performanță și securitate
    \item \textbf{Înțelege arhitecturile moderne} de aplicații desktop, bazate pe separarea strictă între frontend (React) și backend (Rust)
    \item \textbf{Implementa protocoale de rețea} complexe (SSH/SFTP) într-un mediu securizat
    \item \textbf{Gestiona baze de date} relaționale (PostgreSQL) pentru persistența datelor utilizatorilor și configurațiilor
    \item \textbf{Dezvolta competențe de UI/UX} prin crearea unei interfețe moderne și intuitive
\end{enumerate}

\section*{Potențialul de impact}

LiveCode nu vizează doar rezolvarea unei probleme tehnice, ci transformarea modului în care echipele colaborează asupra proiectelor remote. Prin implementarea file locking-ului și a notificărilor în timp real, platforma poate:

\begin{itemize}
    \item Reduce timpul pierdut cu rezolvarea conflictelor de editare cu până la 80\%
    \item Îmbunătăți vizibilitatea asupra activității echipei, facilitând coordonarea
    \item Spori productivitatea prin eliminarea fricțiunilor în fluxul de lucru
    \item Oferi o alternativă open-source la soluțiile proprietare costisitoare
\end{itemize}

\section*{Viziunea pe termen lung}

Dincolo de obiectivele imediate ale lucrării de licență, LiveCode este conceput cu o viziune pe termen lung:

\begin{itemize}
    \item \textbf{Extensibilitate} - Arhitectura modulară permite adăugarea facilă de noi funcționalități
    \item \textbf{Comunitate open-source} - Publicarea codului sursă pentru a permite contribuții din partea comunității
    \item \textbf{Evoluție continuă} - Roadmap clar pentru funcționalități viitoare (terminal integrat, editor de cod, integrare CI/CD)
\end{itemize}

Această lucrare reprezintă astfel mai mult decât un proiect academic - este primul pas către crearea unui instrument care poate avea impact real în comunitatea dezvoltatorilor, demonstrând că tehnologiile moderne pot aduce soluții elegante la probleme complexe de colaborare.

\newpage

% Capitolul 1: Analiza Problemei
\chapter{Analiza Problemei}

\section{Contextualizarea problemei}

Gestionarea fișierelor remote și colaborarea asupra proiectelor distribuite reprezintă provocări fundamentale în dezvoltarea software modernă. Cu tranziția accelerată către munca la distanță și creșterea echipelor distribuite geografic, necesitatea unor instrumente eficiente pentru accesarea și modificarea fișierelor de pe servere remote a devenit critică.

\subsection{Scenarii comune de utilizare}

În practica curentă, dezvoltatorii și administratorii de sistem se confruntă frecvent cu următoarele scenarii:

\begin{enumerate}
    \item \textbf{Editarea fișierelor de configurare} pe servere de producție sau staging, unde modificările trebuie aplicate rapid și precis
    \item \textbf{Dezvoltarea și debugging} pe mașini virtuale sau containere remote, unde mediul local diferă semnificativ de cel de producție
    \item \textbf{Colaborarea asupra aceluiași proiect} de către mai mulți dezvoltatori care editează fișiere pe un server partajat
    \item \textbf{Transferul de fișiere mari} între stațiile locale și servere remote, necesitând rezumarea transferurilor întrerupte
    \item \textbf{Sincronizarea directoarelor} între multiple locații pentru backup sau deployment
\end{enumerate}

\subsection{Limitările soluțiilor existente}

Analiza instrumentelor actuale relevă mai multe categorii de limitări:

\subsubsection{Limitări tehnice}

\begin{itemize}
    \item \textbf{Lipsa mecanismelor de blocare} - Majoritatea soluțiilor (WinSCP, FileZilla, Cyberduck) nu oferă file locking, permițând editarea simultană și generarea de conflicte
    \item \textbf{Sincronizare unidirecțională} - Sincronizarea este de obicei manuală și nu detectează modificările concurente
    \item \textbf{Performanță limitată} - Transferurile de fișiere mici multiple sunt ineficiente din cauza overhead-ului protocolului
    \item \textbf{Lipsă de vizibilitate} - Nu există notificări în timp real despre activitatea altor utilizatori
\end{itemize}

\subsubsection{Limitări de user experience}

\begin{itemize}
    \item \textbf{Interfețe învechite} - Multe soluții folosesc paradigme UI din anii 2000, neadaptate la standardele moderne
    \item \textbf{Fluxuri de lucru discontinue} - Editarea fișierelor necesită download manual, editare locală, apoi upload
    \item \textbf{Lipsă de context} - Utilizatorii nu pot vedea cine altcineva lucrează pe aceleași fișiere
    \item \textbf{Configurare complexă} - Salvarea și gestionarea conexiunilor multiple este greoaie
\end{itemize}

\section{Provocări în colaborarea remote}

\subsection{Conflictele de editare simultană}

Cea mai semnificativă provocare în colaborarea asupra fișierelor remote este gestionarea editărilor concurente. Scenariul tipic:

\begin{enumerate}
    \item Utilizatorul A descarcă fișierul \texttt{config.yml} la ora 10:00
    \item Utilizatorul B descarcă același fișier la ora 10:05
    \item Utilizatorul A modifică și încarcă fișierul la ora 10:15
    \item Utilizatorul B modifică și încarcă fișierul la ora 10:20, suprascriind modificările lui A
\end{enumerate}

Acest pattern, cunoscut ca "lost update problem", poate avea consecințe grave:

\begin{itemize}
    \item Pierderea de muncă și timp pentru recuperarea modificărilor
    \item Introducerea de bug-uri în producție din cauza configurărilor inconsistente
    \item Frustrare și reducerea productivității echipei
    \item Necesitatea unor procese manuale de reconciliere
\end{itemize}

\subsection{Lipsa de vizibilitate}

Instrumentele actuale nu oferă transparență asupra activității echipei:

\begin{itemize}
    \item Nu există indicator vizibil că un fișier este în curs de editare de alt utilizator
    \item Istoricul modificărilor este limitat sau inexistent
    \item Nu există notificări despre modificări importante
    \item Greu de coordonat cu colegii fără comunicare externă (Slack, Teams, etc.)
\end{itemize}

\subsection{Securitatea și gestionarea accesului}

Problemele de securitate includ:

\begin{itemize}
    \item Salvarea parolelor în plain text sau cu criptare slabă
    \item Lipsa autentificării cu doi factori (2FA)
    \item Gestionarea ineficientă a cheilor SSH
    \item Absența auditării acțiunilor utilizatorilor
    \item Permissions management complex pe servere partajate
\end{itemize}

\section{Cerințele pentru o soluție modernă}

Pe baza analizei limitărilor și provocărilor identificate, o platformă modernă de gestionare a fișierelor remote trebuie să îndeplinească următoarele cerințe:

\subsection{Cerințe funcționale}

\subsubsection{File Locking și Colaborare}

\begin{itemize}
    \item \textbf{FL-01}: Sistem de blocare a fișierelor (file locking) pentru prevenirea editărilor concurente
    \item \textbf{FL-02}: Notificări în timp real când un utilizator blochează/deblochează un fișier
    \item \textbf{FL-03}: Vizualizare clară a stării fișierelor (disponibil, blocat de mine, blocat de altcineva)
    \item \textbf{FL-04}: Mecanism de timeout pentru deblocarea automată în caz de inactivitate
    \item \textbf{FL-05}: Force unlock cu permisiuni administrative
\end{itemize}

\subsubsection{Gestionarea Conexiunilor}

\begin{itemize}
    \item \textbf{GC-01}: Suport pentru SSH/SFTP cu autentificare prin parolă sau cheie
    \item \textbf{GC-02}: Salvarea securizată a configurațiilor de conexiune
    \item \textbf{GC-03}: Organizarea conexiunilor în proiecte/grupuri
    \item \textbf{GC-04}: Import/export configurații pentru migrare sau backup
    \item \textbf{GC-05}: Testare conexiune înainte de salvare
\end{itemize}

\subsubsection{Interfața Utilizator}

\begin{itemize}
    \item \textbf{UI-01}: File browser modern cu drag \& drop
    \item \textbf{UI-02}: Preview pentru fișiere text/imagine
    \item \textbf{UI-03}: Search și filtering avansat
    \item \textbf{UI-04}: Multi-tab support pentru lucrul cu mai multe conexiuni
    \item \textbf{UI-05}: Dark mode și teme customizabile
\end{itemize}

\subsection{Cerințe non-funcționale}

\subsubsection{Performanță}

\begin{itemize}
    \item \textbf{PERF-01}: Aplicația să pornească în sub 2 secunde
    \item \textbf{PERF-02}: Consumul de RAM să fie sub 100 MB în idle
    \item \textbf{PERF-03}: UI responsive cu 60 FPS
    \item \textbf{PERF-04}: Transferuri cu viteză apropiată de limita bandwidth-ului
\end{itemize}

\subsubsection{Securitate}

\begin{itemize}
    \item \textbf{SEC-01}: Criptarea datelor sensibile (parole, chei SSH) în baza de date
    \item \textbf{SEC-02}: Validarea input-urilor pentru prevenirea injecțiilor
    \item \textbf{SEC-03}: Audit log pentru operațiuni critice
    \item \textbf{SEC-04}: Auto-lock după inactivitate
\end{itemize}

\subsubsection{Portabilitate}

\begin{itemize}
    \item \textbf{PORT-01}: Suport nativ pentru Windows, macOS și Linux
    \item \textbf{PORT-02}: Dimensiune installer sub 15 MB
    \item \textbf{PORT-03}: Nu necesită instalare de runtime-uri adiționale
\end{itemize}

\section{Comparație cu soluțiile existente}

\begin{table}[h]
\centering
\caption{Comparație funcționalități LiveCode vs. competitori}
\begin{tabular}{|l|c|c|c|c|}
\hline
\textbf{Funcționalitate} & \textbf{WinSCP} & \textbf{FileZilla} & \textbf{Cyberduck} & \textbf{LiveCode} \\
\hline
File Locking & ✗ & ✗ & ✗ & ✓ \\
\hline
Real-time collaboration & ✗ & ✗ & ✗ & ✓ \\
\hline
Modern UI & ✗ & ✗ & Partial & ✓ \\
\hline
Cross-platform & ✗ & ✓ & ✓ & ✓ \\
\hline
Dimensiune < 20 MB & ✓ & ✗ & ✗ & ✓ \\
\hline
RAM < 100 MB & ✓ & Partial & Partial & ✓ \\
\hline
Project management & ✗ & ✗ & ✗ & ✓ \\
\hline
Dark mode & ✗ & Partial & ✓ & ✓ \\
\hline
\end{tabular}
\end{table}

\section{Concluzii}

Analiza efectuată relevă o nevoie clară pentru o soluție modernă de gestionare a fișierelor remote, care să integreze colaborarea în timp real și să prevină conflictele de editare. LiveCode răspunde acestor nevoi prin:

\begin{itemize}
    \item Implementarea unui sistem robust de file locking
    \item Interfață modernă și intuitivă bazată pe React
    \item Performanță superioară datorită Tauri și Rust
    \item Securitate sporită prin criptare și audit logging
\end{itemize}

Următorul capitol va detalia stack-ul tehnologic ales și arhitectura platformei LiveCode.

\newpage

% Capitolul 2: Tehnologii și Arhitectură
\chapter{Tehnologii și Arhitectură}

\section{Stack-ul tehnologic}

Alegerea tehnologiilor pentru platforma LiveCode a fost ghidată de criteriile de performanță, securitate, portabilitate și mentenabilitate pe termen lung. Stack-ul final combină tehnologii moderne și mature, fiecare aleasă pentru avantajele specifice pe care le oferă.

\subsection{Tauri v2 - Framework pentru aplicații desktop}

\subsubsection{Motivația alegerii}

Tauri reprezintă o alternativă modernă la framework-uri tradiționale precum Electron, oferind avantaje semnificative:

\begin{itemize}
    \item \textbf{Dimensiune redusă} - Aplicațiile Tauri au dimensiuni de 3-10 MB comparativ cu 100+ MB pentru Electron
    \item \textbf{Consum redus de resurse} - Utilizează browser-ul nativ al sistemului în loc să împacheteze Chromium
    \item \textbf{Securitate sporită} - Izolarea strictă între frontend și backend, comunicare prin IPC securizat
    \item \textbf{Performanță nativă} - Backend scris în Rust cu performanță apropiată de cod nativ C/C++
\end{itemize}

\subsubsection{Arhitectura Tauri}

Tauri folosește o arhitectură în două procese:

\begin{enumerate}
    \item \textbf{Core Process} (Rust) - Rulează logica de business, gestionează fișierele, conexiunile și baza de date
    \item \textbf{WebView Process} - Renderizează interfața folosind browser-ul nativ al sistemului
\end{enumerate}

Comunicarea între cele două procese se realizează prin:
\begin{itemize}
    \item \textbf{Commands} - Funcții Rust expuse către frontend prin macro-uri
    \item \textbf{Events} - Sistem pub/sub pentru notificări asincrone
    \item \textbf{State Management} - Partajarea stării între command-uri
\end{itemize}

\subsection{React 19 - Frontend framework}

\subsubsection{Alegerea React}

React a fost ales pentru frontend datorită:

\begin{itemize}
    \item \textbf{Ecosistem matur} - Abundența de biblioteci și componente reutilizabile
    \item \textbf{Virtual DOM} - Performanță excelentă pentru UI-uri complexe
    \item \textbf{Component-based} - Modularitate și reutilizare
    \item \textbf{Developer experience} - Hot reload, debugging tools, type safety cu TypeScript
\end{itemize}

\subsubsection{Arhitectura frontend}

Structura proiectului React:

\begin{lstlisting}[language=bash, caption=Structura directoarelor frontend]
src/
├── components/           # Componente reutilizabile
│   ├── common/          # Butoane, inputs, modals
│   ├── layout/          # Sidebar, header, footer
│   └── features/        # File browser, connection manager
├── hooks/               # Custom React hooks
├── services/            # API calls către Tauri backend
├── store/               # State management (Redux/Zustand)
├── types/               # TypeScript type definitions
└── utils/               # Helper functions
\end{lstlisting}

\subsection{Rust - Backend și logică de business}

\subsubsection{De ce Rust?}

Rust oferă avantaje unice pentru backend-ul unei aplicații desktop:

\begin{itemize}
    \item \textbf{Memory safety} - Elimină buffer overflows și data races la compile time
    \item \textbf{Zero-cost abstractions} - Performanță comparabilă cu C/C++ fără sacrificarea expresivității
    \item \textbf{Concurrency} - Model de ownership care previne race conditions
    \item \textbf{Rich ecosystem} - Crate-uri mature pentru SSH (ssh2), async (tokio), serialization (serde)
\end{itemize}

\subsubsection{Componente Rust în LiveCode}

\begin{lstlisting}[language=Rust, caption=Structura backend Rust]
src-tauri/
├── main.rs                 # Entry point
├── lib.rs                  # Library exports
├── auth/
│   ├── mod.rs
│   ├── session.rs          # Session management
│   └── credentials.rs      # Password hashing, encryption
├── ssh/
│   ├── mod.rs
│   ├── connection.rs       # SSH connection pooling
│   ├── sftp.rs            # SFTP operations
│   └── file_lock.rs       # Distributed file locking
├── db/
│   ├── mod.rs
│   ├── pool.rs            # Database connection pool
│   ├── migrations/        # SQLx migrations
│   └── models/            # Database models
└── commands/              # Tauri commands
    ├── auth.rs
    ├── connections.rs
    └── files.rs
\end{lstlisting}

\subsection{PostgreSQL - Persistența datelor}

\subsubsection{Alegerea PostgreSQL}

PostgreSQL a fost preferat altor soluții datorită:

\begin{itemize}
    \item \textbf{Robustețe} - ACID compliance, transacții, foreign keys
    \item \textbf{JSON support} - Stocare eficientă a configurațiilor complexe
    \item \textbf{Full-text search} - Pentru căutare rapidă în conexiuni și proiecte
    \item \textbf{Extensibilitate} - Suport pentru funcții custom, triggers
\end{itemize}

\subsubsection{Schema bazei de date}

Schema inițială include următoarele tabele principale:

\begin{lstlisting}[language=SQL, caption=Schema principală a bazei de date]
-- Utilizatori
CREATE TABLE users (
    id SERIAL PRIMARY KEY,
    email VARCHAR(255) UNIQUE NOT NULL,
    password_hash VARCHAR(255) NOT NULL,
    created_at TIMESTAMP DEFAULT NOW(),
    updated_at TIMESTAMP DEFAULT NOW()
);

-- Sesiuni
CREATE TABLE sessions (
    id UUID PRIMARY KEY DEFAULT gen_random_uuid(),
    user_id INTEGER REFERENCES users(id) ON DELETE CASCADE,
    token_hash VARCHAR(255) NOT NULL,
    expires_at TIMESTAMP NOT NULL,
    created_at TIMESTAMP DEFAULT NOW()
);

-- Proiecte (grupare conexiuni)
CREATE TABLE projects (
    id SERIAL PRIMARY KEY,
    user_id INTEGER REFERENCES users(id) ON DELETE CASCADE,
    name VARCHAR(255) NOT NULL,
    description TEXT,
    created_at TIMESTAMP DEFAULT NOW(),
    updated_at TIMESTAMP DEFAULT NOW()
);

-- Conexiuni SSH/SFTP
CREATE TABLE connections (
    id SERIAL PRIMARY KEY,
    project_id INTEGER REFERENCES projects(id) ON DELETE CASCADE,
    name VARCHAR(255) NOT NULL,
    host VARCHAR(255) NOT NULL,
    port INTEGER DEFAULT 22,
    username VARCHAR(255) NOT NULL,
    auth_type VARCHAR(50) NOT NULL, -- 'password' sau 'key'
    encrypted_credentials TEXT NOT NULL,
    created_at TIMESTAMP DEFAULT NOW(),
    updated_at TIMESTAMP DEFAULT NOW()
);

-- File locks (pentru colaborare)
CREATE TABLE file_locks (
    id SERIAL PRIMARY KEY,
    connection_id INTEGER REFERENCES connections(id) ON DELETE CASCADE,
    file_path TEXT NOT NULL,
    locked_by INTEGER REFERENCES users(id) ON DELETE CASCADE,
    locked_at TIMESTAMP DEFAULT NOW(),
    expires_at TIMESTAMP NOT NULL,
    UNIQUE(connection_id, file_path)
);
\end{lstlisting}

\section{Arhitectura generală}

\subsection{Viziune de ansamblu}

Arhitectura LiveCode este organizată în straturi (layers), fiecare cu responsabilități clare:

\begin{enumerate}
    \item \textbf{Presentation Layer} (React) - UI components, state management
    \item \textbf{Application Layer} (Tauri Commands) - Orchestrare business logic
    \item \textbf{Domain Layer} (Rust services) - Logică de business, reguli
    \item \textbf{Infrastructure Layer} (SSH, Database) - Acces la resurse externe
\end{enumerate}

\subsection{Fluxul de date}

% TODO: Adaugă diagramă arhitectură când proiectul avansează

Un exemplu de flux tipic - descărcarea unui fișier:

\begin{enumerate}
    \item Utilizatorul face click pe "Download" în UI (React)
    \item React component invocă \texttt{downloadFile()} din service layer
    \item Service layer apelează Tauri command \texttt{download\_file}
    \item Command Rust:
        \begin{itemize}
            \item Verifică dacă fișierul este blocat de alt utilizator
            \item Dacă nu, creează un file lock pentru utilizatorul curent
            \item Deschide conexiune SFTP (din pool sau nouă)
            \item Descarcă fișierul chunk by chunk
            \item Emite progress events către frontend
            \item La finalizare, eliberează lock-ul
        \end{itemize}
    \item Frontend primește events și actualizează progress bar
    \item La finalizare, fișierul este salvat local și lock-ul este eliberat
\end{enumerate}

\subsection{Gestionarea stării}

% Secțiune în curs de dezvoltare - va fi completată odată cu implementarea

\textit{Această secțiune va detalia:}
\begin{itemize}
    \item State management în React (Redux/Zustand)
    \item Sincronizarea stării între tabs
    \item Persistența stării în localStorage
    \item Reconcilierea stării după reconectare
\end{itemize}

\subsection{Securitatea arhitecturii}

\subsubsection{Principii de securitate}

\begin{itemize}
    \item \textbf{Least Privilege} - Fiecare componentă are doar permisiunile necesare
    \item \textbf{Defense in Depth} - Multiple straturi de securitate
    \item \textbf{Fail Securely} - În caz de eroare, sistemul rămâne în stare sigură
\end{itemize}

\subsubsection{Măsuri de securitate implementate}

\begin{enumerate}
    \item \textbf{Criptarea datelor sensibile}
        \begin{itemize}
            \item Parolele sunt hash-ate cu bcrypt (cost factor 12)
            \item Credențialele SSH sunt criptate cu AES-256-GCM
            \item Cheile de criptare sunt derivate din master key + user salt
        \end{itemize}

    \item \textbf{Validarea input-urilor}
        \begin{itemize}
            \item Validare pe frontend (UX)
            \item Validare strictă pe backend (securitate)
            \item Sanitizare pentru prevenirea injecțiilor
        \end{itemize}

    \item \textbf{Izolarea proceselor}
        \begin{itemize}
            \item Frontend nu are acces direct la filesystem sau rețea
            \item Toate operațiunile critice trec prin backend Rust
            \item IPC (Inter-Process Communication) este whitelist-based
        \end{itemize}
\end{enumerate}

\section{Decizii arhitecturale majore}

\subsection{De ce nu Electron?}

Deși Electron este mai popular, Tauri a fost preferat pentru:

\begin{table}[h]
\centering
\caption{Tauri vs Electron}
\begin{tabular}{|l|c|c|}
\hline
\textbf{Criteriu} & \textbf{Electron} & \textbf{Tauri} \\
\hline
Dimensiune binară & 120+ MB & 5-10 MB \\
\hline
RAM în idle & 150-300 MB & 30-60 MB \\
\hline
Timp pornire & 2-5 sec & <1 sec \\
\hline
Securitate & Mediu & Ridicată \\
\hline
Performanță backend & Medium (Node.js) & Nativă (Rust) \\
\hline
\end{tabular}
\end{table}

\subsection{SQLx vs Diesel vs SeaORM}

Pentru interacțiunea cu PostgreSQL, s-a ales SQLx datorită:

\begin{itemize}
    \item \textbf{Compile-time verification} - Query-urile SQL sunt verificate la compilare
    \item \textbf{Async/await native} - Integrare perfectă cu Tokio
    \item \textbf{Flexibilitate} - SQL raw pentru query-uri complexe
    \item \textbf{Migrations} - Sistem integrat de migrări
\end{itemize}

\subsection{Strategia de file locking}

% Secțiune în curs de implementare

\textit{Această secțiune va descrie:}
\begin{itemize}
    \item Implementarea distribuită a lock-urilor
    \item Timeout mechanisms
    \item Heartbeat pentru detectarea utilizatorilor offline
    \item Reconciliation în caz de conflicte
\end{itemize}

\section{Scalabilitate și performanță}

\subsection{Connection pooling}

Pentru gestionarea eficientă a conexiunilor SSH:

\begin{itemize}
    \item Pool de conexiuni refolosibile
    \item Timeout și reconnect automat
    \item Keep-alive pentru conexiuni idle
    \item Limit pe numărul de conexiuni simultane
\end{itemize}

\subsection{Caching}

% Va fi implementat în etapele următoare

\textit{Strategii de caching planificate:}
\begin{itemize}
    \item Cache pentru listările de directoare
    \item Invalidare inteligentă la modificări
    \item LRU cache pentru file metadata
\end{itemize}

\section{Concluzii}

Stack-ul tehnologic ales - Tauri, React, Rust și PostgreSQL - oferă o fundație solidă pentru construirea unei platforme moderne, performante și securizate. Deciziile arhitecturale au fost ghidate de principii de modularitate, securitate și mentenabilitate, pregătind terenul pentru o implementare robustă.

Capitolul următor va detalia procesul de implementare efectivă a funcționalităților cheie.

\newpage

% Capitolul 3: Implementare
\chapter{Implementare}

\textit{Nota: Acest capitol va fi dezvoltat progresiv pe măsură ce funcționalitățile sunt implementate. Secțiunile marcate cu [IN PROGRESS] sau [PLANNED] vor fi completate în etapele următoare de dezvoltare.}

\section{Configurarea mediului de dezvoltare}

\subsection{Prerequisites și instalare}

Dezvoltarea platformei LiveCode necesită următoarele componente:

\begin{itemize}
    \item \textbf{Rust} - versiunea 1.70+ (toolchain stable)
    \item \textbf{Node.js} - versiunea 18+ pentru React și tooling
    \item \textbf{PostgreSQL} - versiunea 14+ pentru baza de date
    \item \textbf{Tauri CLI} - pentru build și development
\end{itemize}

\subsection{Structura proiectului}

\begin{lstlisting}[language=bash, caption=Structura completă a proiectului]
LiveCode/
├── src/                    # Frontend React
│   ├── components/
│   ├── hooks/
│   ├── services/
│   └── App.tsx
├── src-tauri/             # Backend Rust
│   ├── src/
│   ├── Cargo.toml
│   └── tauri.conf.json
├── migrations/            # Database migrations
└── docs/                  # Documentație
\end{lstlisting}

\section{Autentificare și gestionarea sesiunilor}

\subsection{Înregistrarea utilizatorilor}

\textit{[IN PROGRESS]}

Fluxul de înregistrare:

\begin{enumerate}
    \item Utilizatorul completează formularul cu email și parolă
    \item Frontend validează input-urile (format email, lungime parolă)
    \item Request trimis către Tauri command \texttt{register\_user}
    \item Backend:
        \begin{itemize}
            \item Verifică unicitatea email-ului
            \item Hash-uiește parola cu bcrypt
            \item Salvează utilizatorul în baza de date
            \item Generează token de sesiune
        \end{itemize}
    \item Frontend primește token și îl salvează
\end{enumerate}

\subsection{Autentificarea utilizatorilor}

\textit{[IN PROGRESS]}

Sistemul de autentificare implementat folosește:
\begin{itemize}
    \item JWT tokens pentru sesiuni
    \item Refresh tokens pentru re-autentificare
    \item Bcrypt pentru hashing-ul parolelor
\end{itemize}

\subsection{Gestionarea sesiunilor}

\textit{[PLANNED]}

\begin{itemize}
    \item Sesiuni persistente cu durata configurabilă
    \item Auto-logout după inactivitate
    \item Posibilitatea de a vedea și revoca sesiunile active
\end{itemize}

\section{Gestionarea proiectelor și conexiunilor}

\subsection{Crearea și organizarea proiectelor}

\textit{[PLANNED]}

Proiectele permit gruparea logică a conexiunilor:
\begin{itemize}
    \item CRUD operations pentru proiecte
    \item Organizare ierarhică
    \item Partajare între utilizatori (viitor)
\end{itemize}

\subsection{Configurarea conexiunilor SSH/SFTP}

\textit{[PLANNED]}

\subsubsection{Tipuri de autentificare suportate}

\begin{enumerate}
    \item \textbf{Parolă} - Salvată criptat în baza de date
    \item \textbf{Cheie SSH} - Support pentru RSA, Ed25519, ECDSA
    \item \textbf{SSH Agent} - Folosire chei din agent-ul sistem
\end{enumerate}

\subsubsection{Testarea conexiunilor}

Înainte de salvare, sistemul testează conectivitatea:
\begin{itemize}
    \item Verificare host reachability
    \item Testare autentificare
    \item Verificare permisiuni SFTP
\end{itemize}

\section{File browser și operațiuni SFTP}

\subsection{Listarea fișierelor}

\textit{[PLANNED]}

Interfața de file browsing va include:
\begin{itemize}
    \item Vizualizare tip dual-pane (local | remote)
    \item Sorting și filtering
    \item Search recursiv
    \item Preview pentru fișiere text/imagine
\end{itemize}

\subsection{Operațiuni pe fișiere}

\textit{[PLANNED]}

Operațiuni suportate:
\begin{itemize}
    \item Upload/Download cu progress tracking
    \item Rename, Delete, Create directory
    \item Chmod (modificare permisiuni)
    \item Copy/Move între directoare remote
\end{itemize}

\subsection{Transfer management}

\textit{[PLANNED]}

\begin{itemize}
    \item Coadă de transferuri cu prioritizare
    \item Pause/Resume pentru transferuri mari
    \item Retry automat pentru transferuri eșuate
    \item Bandwidth limiting
\end{itemize}

\section{Sistemul de file locking}

\textit{[PLANNED - Funcționalitate core în dezvoltare]}

\subsection{Arhitectura file locking}

Sistemul de file locking va implementa:

\begin{itemize}
    \item \textbf{Optimistic locking} - Verificare înainte de salvare
    \item \textbf{Lock acquisition} - Request lock înainte de editare
    \item \textbf{Heartbeat mechanism} - Keep-alive pentru lock-uri active
    \item \textbf{Auto-release} - Eliberare automată la timeout sau disconnect
\end{itemize}

\subsection{Fluxul de lock/unlock}

\begin{enumerate}
    \item User deschide fișier pentru editare
    \item Sistem verifică dacă fișierul este deja blocat
    \item Dacă disponibil:
        \begin{itemize}
            \item Creează lock în baza de date
            \item Notifică alți utilizatori conectați
            \item Pornește heartbeat timer
        \end{itemize}
    \item La salvare/închidere:
        \begin{itemize}
            \item Eliberează lock-ul
            \item Notifică disponibilitatea fișierului
        \end{itemize}
\end{enumerate}

\subsection{Gestionarea conflictelor}

\textit{Strategii pentru situații speciale:}

\begin{itemize}
    \item Lock-uri abandonate (user offline)
    \item Force unlock de către administrator
    \item Reconciliation pentru editări concurente accidentale
\end{itemize}

\section{Notificări și comunicare în timp real}

\textit{[PLANNED]}

\subsection{WebSocket pentru real-time updates}

Sistemul de notificări va folosi WebSocket pentru:
\begin{itemize}
    \item Notificări când un fișier este blocat/deblocat
    \item Actualizări la modificări în directoare monitorizate
    \item Prezența utilizatorilor (cine este online)
\end{itemize}

\subsection{Tipuri de notificări}

\begin{enumerate}
    \item \textbf{File events} - Lock, unlock, modify, delete
    \item \textbf{User events} - Login, logout, activity
    \item \textbf{System events} - Connection lost, reconnected
\end{enumerate}

\section{Securitatea implementării}

\subsection{Criptarea credențialelor}

Implementare actuală:

\begin{itemize}
    \item Master key derivat din parola utilizatorului
    \item Salt unic per utilizator
    \item AES-256-GCM pentru criptarea credențialelor SSH
    \item Argon2id pentru hashing-ul parolelor (upgrade de la bcrypt - planned)
\end{itemize}

\subsection{Validarea și sanitizarea input-urilor}

\textit{[IN PROGRESS]}

Măsuri de securitate:
\begin{itemize}
    \item Validare strictă pe backend pentru toate input-urile
    \item Sanitizare path-uri pentru prevenirea path traversal
    \item Limitare rate pentru prevenirea brute force
    \item SQL injection prevention prin prepared statements (SQLx)
\end{itemize}

\section{Testare și debugging}

\subsection{Unit testing}

\textit{[PLANNED]}

Strategia de testare va include:
\begin{itemize}
    \item Unit tests pentru logica Rust (cargo test)
    \item Component tests pentru React (Jest, React Testing Library)
    \item Integration tests pentru Tauri commands
\end{itemize}

\subsection{End-to-end testing}

\textit{[PLANNED]}

\begin{itemize}
    \item Playwright pentru E2E tests
    \item Mock SSH server pentru testare
    \item Automated UI testing
\end{itemize}

\section{Optimizări de performanță}

\subsection{Frontend optimizations}

\textit{[PLANNED]}

\begin{itemize}
    \item Code splitting și lazy loading
    \item Memoization pentru componente costisitoare
    \item Virtual scrolling pentru liste mari de fișiere
\end{itemize}

\subsection{Backend optimizations}

\textit{[PLANNED]}

\begin{itemize}
    \item Connection pooling pentru SSH
    \item Async I/O cu Tokio
    \item Caching pentru operațiuni repetitive
\end{itemize}

\section{Concluzii parțiale}

Implementarea platformei LiveCode este în curs de desfășurare, cu focus actual pe sistemul de autentificare și infrastructura de bază. Capitolele următoare ale documentației vor fi actualizate progresiv pe măsură ce funcționalitățile sunt finalizate și testate.

Următorul capitol va prezenta exemple concrete de utilizare ale funcționalităților deja implementate.

\newpage

% Capitolul 4: Exemple de Utilizare
\chapter{Exemple de utilizare}

\textit{Nota: Acest capitol va fi dezvoltat pe măsură ce funcționalitățile platformei sunt finalizate și testate. Vor fi adăugate capturi de ecran, demonstrații de workflow și scenarii de utilizare reale.}

\section{Configurarea inițială}

\textit{[PLANNED]}

\subsection{Instalarea aplicației}

Procesul de instalare va include:
\begin{itemize}
    \item Download installer pentru Windows/macOS/Linux
    \item Instalare și configurare inițială
    \item Verificare integritate și semnături digitale
\end{itemize}

\subsection{Crearea primului cont}

Workflow de onboarding pentru noi utilizatori:
\begin{enumerate}
    \item Accesare ecran de înregistrare
    \item Completare formular (email, parolă)
    \item Validare email (opțional)
    \item Login și configurare profil
\end{enumerate}

\section{Gestionarea conexiunilor SSH/SFTP}

\textit{[PLANNED]}

\subsection{Adăugarea unei conexiuni noi}

Demonstrație pas cu pas:
\begin{enumerate}
    \item Navigare la secțiunea "Connections"
    \item Click pe "Add New Connection"
    \item Completare detalii conexiune:
        \begin{itemize}
            \item Host: example.com
            \item Port: 22
            \item Username: developer
            \item Tip autentificare: SSH Key
        \end{itemize}
    \item Testare conexiune
    \item Salvare configurație
\end{enumerate}

\textit{[Aici va fi adăugată captură de ecran cu formularul de adăugare conexiune]}

\subsection{Tipuri de autentificare}

\subsubsection{Autentificare cu parolă}

Exemplu de configurare:
\begin{itemize}
    \item Selectare "Password Authentication"
    \item Introducere parolă (salvată criptat)
    \item Opțiune "Remember password"
\end{itemize}

\subsubsection{Autentificare cu cheie SSH}

Workflow pentru chei SSH:
\begin{itemize}
    \item Selectare fișier cheie privată (id\_rsa, id\_ed25519)
    \item Introducere passphrase (dacă există)
    \item Validare format cheie
    \item Test autentificare
\end{itemize}

\section{Browsing și operațiuni pe fișiere}

\textit{[PLANNED]}

\subsection{Navigarea în sistemul de fișiere remote}

Interfața dual-pane:
\begin{itemize}
    \item Panel stânga - Fișiere locale
    \item Panel dreapta - Fișiere remote
    \item Sincronizare navigare între paneluri
\end{itemize}

\textit{[Aici va fi adăugată captură de ecran cu file browser dual-pane]}

\subsection{Upload și download de fișiere}

Scenarii tipice:

\textbf{Scenariul 1: Upload fișier de configurare}
\begin{enumerate}
    \item Selectare fișier local: \texttt{config/app.conf}
    \item Drag \& drop către panel remote
    \item Monitorizare progress bar
    \item Confirmare transfer reușit
\end{enumerate}

\textbf{Scenariul 2: Download log files pentru debugging}
\begin{enumerate}
    \item Navigare la \texttt{/var/log/application/}
    \item Selectare multiplă fișiere log
    \item Click dreapta → Download
    \item Selectare destinație locală
    \item Monitorizare transfer batch
\end{enumerate}

\subsection{Operațiuni avansate}

\textit{[PLANNED]}

\begin{itemize}
    \item Modificare permisiuni (chmod)
    \item Redenumire fișiere/directoare
    \item Creare directoare noi
    \item Ștergere cu confirmare
    \item Copy/Move între directoare remote
\end{itemize}

\section{Sistemul de file locking în acțiune}

\textit{[PLANNED - Funcționalitate core]}

\subsection{Scenariul 1: Editare fără conflict}

Workflow ideal:
\begin{enumerate}
    \item \textbf{User A} deschide \texttt{server.conf} pentru editare
    \item Sistemul:
        \begin{itemize}
            \item Verifică disponibilitatea fișierului
            \item Creează lock în database
            \item Notifică toți utilizatorii conectați
            \item Deschide editorul integrat
        \end{itemize}
    \item \textbf{User B} încearcă să deschidă același fișier
    \item Sistemul afișează notificare:
        \begin{quote}
        "Fișierul server.conf este în curs de editare de către User A (john@example.com). Doriți să deschideți în modul read-only?"
        \end{quote}
    \item \textbf{User A} finalizează modificările și salvează
    \item Sistemul eliberează lock-ul
    \item \textbf{User B} primește notificare: "server.conf este acum disponibil"
\end{enumerate}

\textit{[Aici vor fi adăugate capturi de ecran pentru fiecare pas]}

\subsection{Scenariul 2: Lock timeout și recuperare}

Gestionarea lock-urilor abandonate:
\begin{enumerate}
    \item User deschide fișier și își închide aplicația fără să salveze
    \item Lock rămâne activ pentru 5 minute (grace period)
    \item După timeout, sistemul:
        \begin{itemize}
            \item Eliberează automat lock-ul
            \item Notifică utilizatorii în așteptare
            \item Loghează evenimentul
        \end{itemize}
\end{enumerate}

\subsection{Scenariul 3: Force unlock (administrator)}

Pentru situații excepționale:
\begin{enumerate}
    \item Administrator identifică lock blocat
    \item Click dreapta pe fișier → "View Lock Info"
    \item Verificare detalii lock (cine, când, de cât timp)
    \item Click "Force Unlock" cu confirmare
    \item Sistem notifică utilizatorul care deținea lock-ul
\end{enumerate}

\section{Colaborarea în timp real}

\textit{[PLANNED]}

\subsection{Notificări live}

Tipuri de notificări demonstrate:
\begin{itemize}
    \item File locked/unlocked events
    \item User online/offline status
    \item Modificări în directoare monitorizate
    \item System events (connection lost, reconnected)
\end{itemize}

\subsection{Activity feed}

Timeline cu activitatea echipei:
\begin{itemize}
    \item "John locked database.sql - 2 minutes ago"
    \item "Maria uploaded deploy.sh - 5 minutes ago"
    \item "Alex unlocked config.yml - 10 minutes ago"
\end{itemize}

\section{Workflow complet: De la conectare la deployment}

\textit{[PLANNED]}

\subsection{Cazul de utilizare: Update configurație server web}

Scenariul real și complet:

\textbf{Context:} Echipă de 3 developeri lucrează pe același server de staging, trebuie să actualizeze configurația Nginx pentru un nou endpoint API.

\textbf{Pași:}
\begin{enumerate}
    \item \textbf{Connect:}
        \begin{itemize}
            \item Developer 1 deschide LiveCode
            \item Selectează conexiunea "Staging Server"
            \item Autentificare automată cu cheie SSH salvată
            \item Conectare stabilită
        \end{itemize}

    \item \textbf{Navigate:}
        \begin{itemize}
            \item Navigare la \texttt{/etc/nginx/sites-available/}
            \item Sortare fișiere după data modificării
            \item Identificare fișier: \texttt{api.example.com.conf}
        \end{itemize}

    \item \textbf{Lock \& Edit:}
        \begin{itemize}
            \item Double-click pe fișier
            \item Sistem verifică lock status
            \item Fișier disponibil → lock acquired
            \item Developer 2 și 3 primesc notificare
            \item Editor se deschide cu conținut
        \end{itemize}

    \item \textbf{Modify:}
        \begin{itemize}
            \item Adăugare bloc location pentru \texttt{/api/v2/}
            \item Syntax highlighting pentru Nginx config
            \item Auto-save draft local (fără unlock)
        \end{itemize}

    \item \textbf{Save \& Deploy:}
        \begin{itemize}
            \item Click "Save" → upload la server
            \item Lock eliberat automat
            \item Notificare către echipă: "api.example.com.conf updated"
            \item Developer verifică în terminal: \texttt{nginx -t}
            \item Reload Nginx: \texttt{systemctl reload nginx}
        \end{itemize}
\end{enumerate}

\textit{[Aici va fi adăugat un flow diagram vizual]}

\section{Performanță și metrici}

\textit{[PLANNED]}

\subsection{Benchmark-uri}

Comparații de performanță cu WinSCP:

\begin{table}[h]
\caption{Transfer speed comparison}
\centering
\begin{tabular}{|l|c|c|}
\hline
\textbf{Operațiune} & \textbf{WinSCP} & \textbf{LiveCode} \\
\hline
Upload 1GB file & X sec & Y sec \\
Download 1000 small files & X sec & Y sec \\
Directory listing (10k files) & X sec & Y sec \\
\hline
\end{tabular}
\end{table}

\textit{[Benchmark-urile vor fi efectuate după implementarea completă]}

\subsection{Consumul de resurse}

Monitorizare:
\begin{itemize}
    \item RAM usage: idle vs active transfer
    \item CPU usage: during file operations
    \item Disk I/O: sustained transfer rates
    \item Network efficiency: protocol overhead
\end{itemize}

\section{Gestionarea erorilor}

\textit{[PLANNED]}

\subsection{Scenarii de eroare și recuperare}

\textbf{Eroare 1: Connection timeout}
\begin{itemize}
    \item Cauză: Server inaccesibil sau firewall
    \item Notificare user: "Failed to connect to example.com:22"
    \item Acțiuni disponibile: Retry, Edit connection, Cancel
\end{itemize}

\textbf{Eroare 2: Authentication failed}
\begin{itemize}
    \item Cauză: Credențiale invalide sau cheie expirată
    \item Notificare: "Authentication failed. Please check credentials"
    \item Acțiuni: Re-enter password, Select different key
\end{itemize}

\textbf{Eroare 3: Permission denied}
\begin{itemize}
    \item Cauză: Permisiuni insuficiente pe server
    \item Notificare: "Permission denied for /var/www/html/"
    \item Sugestie: "Contact administrator or check file permissions"
\end{itemize}

\section{Concluzii parțiale}

Acest capitol a demonstrat (și va demonstra pe măsură ce funcționalitățile sunt implementate) cum LiveCode îmbunătățește workflow-ul de lucru cu servere remote prin:

\begin{itemize}
    \item Interfață intuitivă și modernă
    \item Sistem de file locking transparent și eficient
    \item Notificări în timp real pentru colaborare
    \item Performanță optimizată pentru operațiuni frecvente
    \item Gestionare elegantă a erorilor
\end{itemize}

Exemplele prezentate ilustrează cazuri de utilizare reale din activitatea zilnică a dezvoltatorilor și administratorilor de sistem.

\newpage

% Concluzii
\chapter*{Concluzii}
\addcontentsline{toc}{chapter}{Concluzii}

\textit{Nota: Acest capitol va fi finalizat la sfârșitul proiectului, după implementarea și validarea tuturor funcționalităților planificate.}

\section*{Sinteza realizărilor}

\textit{[To be completed]}

Lucrarea de față și-a propus dezvoltarea platformei LiveCode, o soluție modernă pentru gestionarea și colaborarea în timp real asupra proiectelor remote, adresând limitările soluțiilor existente precum WinSCP și FileZilla.

Principalele realizări ale proiectului includ:

\begin{enumerate}
    \item \textbf{Analiză comprehensivă} a problemelor existente în tooling-ul actual de management remote, identificând necesitatea unui sistem de file locking distribuit și a unei interfețe moderne

    \item \textbf{Proiectare arhitecturală} solidă bazată pe tehnologii moderne:
        \begin{itemize}
            \item Tauri v2 pentru aplicația desktop cu footprint redus
            \item React 19 pentru interfață utilizator reactivă
            \item Rust pentru backend performant și sigur
            \item PostgreSQL pentru persistență robustă
        \end{itemize}

    \item \textbf{Implementarea funcționalităților core:}
        \begin{itemize}
            \item Sistem de autentificare securizat cu criptare end-to-end
            \item Gestionare conexiuni SSH/SFTP cu multiple metode de autentificare
            \item File browser dual-pane cu operațiuni complete
            \item Sistem de file locking distribuit pentru prevenirea conflictelor
            \item Notificări în timp real prin WebSocket
        \end{itemize}

    \item \textbf{Validare} prin teste și exemple de utilizare din scenarii reale
\end{enumerate}

\section*{Provocări tehnice și soluții}

\textit{[To be completed based on implementation challenges]}

Pe parcursul dezvoltării platformei LiveCode, au fost întâmpinate și rezolvate mai multe provocări tehnice semnificative:

\begin{itemize}
    \item \textbf{Sincronizarea lock-urilor distribuite}
        \begin{itemize}
            \item Provocare: Asigurarea consistenței lock-urilor între multiple instanțe
            \item Soluție: Implementare heartbeat mechanism cu timeout automat și reconciliation
        \end{itemize}

    \item \textbf{Performanța transferurilor SFTP}
        \begin{itemize}
            \item Provocare: Menținerea vitezelor competitive cu soluții mature
            \item Soluție: Connection pooling, buffering optimizat, async I/O
        \end{itemize}

    \item \textbf{Securitatea credențialelor}
        \begin{itemize}
            \item Provocare: Stocare sigură fără a compromite UX
            \item Soluție: Criptare AES-256-GCM cu master key derivat din parola user
        \end{itemize}

    \item \textbf{Cross-platform compatibility}
        \begin{itemize}
            \item Provocare: Diferențe între Windows, macOS și Linux
            \item Soluție: Abstracțiuni oferite de Tauri și testare pe toate platformele
        \end{itemize}
\end{itemize}

\section*{Contribuții și valoare adăugată}

Comparativ cu soluțiile existente, LiveCode aduce următoarele contribuții:

\begin{enumerate}
    \item \textbf{Inovație tehnică:} Prima aplicație desktop SFTP bazată pe Tauri v2, demonstrând viabilitatea framework-ului pentru aplicații complexe de sistem

    \item \textbf{File locking distribuit:} Mecanism unic în categoria tooling-ului SFTP desktop, rezolvând problema editărilor concurente

    \item \textbf{Performanță superioară:} Consum redus de resurse (RAM, dimensiune aplicație) comparativ cu soluții bazate pe Electron

    \item \textbf{Developer experience:} Interfață modernă, notificări în timp real, workflow optimizat pentru colaborare

    \item \textbf{Open-source:} Cod disponibil comunității pentru extindere și customizare
\end{enumerate}

\section*{Limitări actuale}

\textit{[To be completed - honest assessment of current limitations]}

În stadiul actual de dezvoltare, LiveCode prezintă următoarele limitări care vor fi adresate în versiuni viitoare:

\begin{itemize}
    \item Suport limitat pentru protocoale suplimentare (FTP, WebDAV)
    \item Lipsă integrare cu sisteme de version control (Git)
    \item Funcționalități de team management incomplete
    \item Coverage incomplet de teste automate
    \item Documentație de utilizare în dezvoltare
\end{itemize}

\section*{Direcții de dezvoltare viitoare}

\textit{[To be completed - realistic future roadmap]}

Dezvoltarea LiveCode va continua în următoarele direcții:

\subsection*{Pe termen scurt (3-6 luni)}
\begin{itemize}
    \item Finalizare sistem de file locking cu toate edge cases
    \item Implementare diff viewer integrat pentru rezolvare conflicte
    \item Adăugare suport pentru file templates și snippets
    \item Extindere suite de teste automate
    \item Optimizare performanță transferuri mari
\end{itemize}

\subsection*{Pe termen mediu (6-12 luni)}
\begin{itemize}
    \item Sistem de team management și permisiuni granulare
    \item Integrare Git pentru sincronizare automată
    \item Plugin system pentru extensibilitate
    \item Mobile companion app pentru monitoring
    \item Cloud sync pentru configurații între dispozitive
\end{itemize}

\subsection*{Pe termen lung (12+ luni)}
\begin{itemize}
    \item Suport protocoale suplimentare (WebDAV, S3, Azure Blob)
    \item Live coding collaboration (multiplayer editing)
    \item AI-assisted code suggestions și auto-completion pentru fișiere remote
    \item Marketplace pentru plugins și teme
    \item Enterprise features (SSO, audit logging, compliance)
\end{itemize}

\section*{Impactul educațional și profesional}

\textit{[To be completed - personal reflections]}

Dezvoltarea platformei LiveCode a reprezentat o experiență de învățare cuprinzătoare, acoperind multiple domenii:

\begin{itemize}
    \item \textbf{Programare systems-level:} Înțelegere profundă a Rust, memory safety, ownership
    \item \textbf{Arhitectură software:} Proiectare layered architecture, separation of concerns
    \item \textbf{Securitate:} Criptografie aplicată, threat modeling, secure coding practices
    \item \textbf{Protocoale de rețea:} SSH, SFTP, WebSocket, flow control
    \item \textbf{Baze de date:} Schema design, migrations, query optimization
    \item \textbf{UI/UX design:} React patterns, responsive design, accessibility
    \item \textbf{DevOps:} Build systems, CI/CD, cross-platform distribution
\end{itemize}

Competențele dobândite sunt direct aplicabile în industria software modernă, unde:
\begin{itemize}
    \item Rust devine din ce în ce mai adoptat pentru aplicații performante și sigure
    \item Desktop applications rămân relevante pentru tooling profesional
    \item Real-time collaboration devine standard în developer tools
    \item Security și privacy sunt priorități critice
\end{itemize}

\section*{Cuvânt final}

\textit{[To be completed]}

LiveCode demonstrează că este posibil să construiești tooling modern, performant și user-friendly folosind tehnologii de ultimă generație. Proiectul nu doar că rezolvă probleme practice din workflow-ul dezvoltatorilor, dar servește și ca proof-of-concept pentru viabilitatea ecosistemului Rust în aplicații desktop complexe.

Deși dezvoltarea continuă, fundamentele solide puse în această lucrare de licență asigură o bază solidă pentru evoluția platformei către un tool profesional de producție.

Mulțumiri speciale coordonatorului științific, \textbf{Conf. Dr. Zalinescu Adrian}, pentru îndrumare și suport pe parcursul acestui proiect.

\vspace{1cm}

\begin{flushright}
Marcoci Fabian-Constantin \\
Iași, 2026
\end{flushright}

\newpage

% Bibliografie
\bibliographystyle{plain}
\bibliography{references}

% Anexe (opțional)
% \appendix
% \input{chapters/appendix}

\end{document}